\documentclass[10pt,landscape]{article}
\usepackage{amssymb,amsmath,amsthm,amsfonts}
\usepackage{multicol,multirow}
\usepackage{calc}
\usepackage{ifthen}
\usepackage[landscape]{geometry}
\usepackage[colorlinks=true,citecolor=blue,linkcolor=blue]{hyperref}


\ifthenelse{\lengthtest { \paperwidth = 11in}}
    { \geometry{top=.5in,left=.5in,right=.5in,bottom=.5in} }
	{\ifthenelse{ \lengthtest{ \paperwidth = 297mm}}A
		{\geometry{top=1cm,left=1cm,right=1cm,bottom=1cm} }
		{\geometry{top=1cm,left=1cm,right=1cm,bottom=1cm} }
	}
\pagestyle{empty}
\makeatletter
\renewcommand{\section}{\@startsection{section}{1}{0mm}%
                                {-1ex plus -.5ex minus -.2ex}%
                                {0.5ex plus .2ex}%x
                                {\normalfont\large\bfseries}}
\renewcommand{\subsection}{\@startsection{subsection}{2}{0mm}%
                                {-1explus -.5ex minus -.2ex}%
                                {0.5ex plus .2ex}%
                                {\normalfont\normalsize\bfseries}}
\renewcommand{\subsubsection}{\@startsection{subsubsection}{3}{0mm}%
                                {-1ex plus -.5ex minus -.2ex}%
                                {1ex plus .2ex}%
                                {\normalfont\small\bfseries}}
\makeatother
\setcounter{secnumdepth}{0}
\setlength{\parindent}{0pt}
\setlength{\parskip}{0pt plus 0.5ex}
% -----------------------------------------------------------------------

\title{Raketentreibstoffe I+II - Cheat Sheet}

\begin{document}

\raggedright
\footnotesize

\begin{center}
     \Large{\textbf{Raketentreibstoffe I+II - Cheat Sheet}} \\
\end{center}
\begin{multicols}{3}
\setlength{\premulticols}{1pt}
\setlength{\postmulticols}{1pt}
\setlength{\multicolsep}{1pt}
\setlength{\columnsep}{2pt}

\section{Basics}
Most chemical equations are not valid at high temperatures because of dissociation effects. More dissociation will result in a less released heat during the reaction since energy is needed to break the molecules apart. In general, the following statements are valid during combustion reactions.\\
\vspace{5pt}
\textbf{Higher} pressures lead to \textbf{less} dissociation.\\
\textbf{Higher} temperatures lead to \textbf{more} dissociation.\\
\vspace{7pt}
\begin{tabular}{lll}
	Thrust & $F$ & N \\
	Chamber pressure & $p_c$ & Pa \\
	Nozzle exit pressure & $p_e$ & Pa \\
	Ambient pressure & $p_0$ & Pa \\
	Oxidizer fuel ratio & ROF & - \\
	Initial mass & $m_0$ & kg \\
	Burnout mass & $m_b$ & kg \\
\end{tabular}\\
\vspace{5pt}
Ziolkowski equation $\Delta v=I_{\text{sp}}\cdot g_0 \cdot \text{ln}(\frac{m_0}{m_b})$\\
\vspace{7pt}
Total impulse $I_T = \int_{0}^{t_e} F dt$\\
Specific vacuum impulse $I_{\text{sp}}=\frac{F}{\dot{m}g}$\\
\vspace{7pt}
Thrust equation $F=\dot{m}\cdot c_e + A_e \cdot (p_e - p_0)$\\
Thrust coefficient $c_F = \frac{F}{A_t p_c}$\\
\vspace{7pt}
Characteristic length $L^*=V_c/A_t$\\
Characteristic speed $c^*=\frac{p_cA_t}{\dot{m}}$\\
Characteristic speed  $c^*=\frac{\sqrt{\kappa R T_c}}{\kappa \sqrt{(\frac{2}{\kappa+1})^\frac{\kappa+1}{\kappa-1}}}$\\
Efficiency of combustion $\eta_{c^*} = \frac{c^*_{\text{real}}}{c^*}$

\section{Thermodynamics}
\begin{tabular}{lll}
	Degrees of Molecular Freedom & $f$ & - \\
\end{tabular}\\
\vspace{5pt}
Ideal gas equation $pV = mRT$\\
\vspace{5pt}
Isentropic Coefficient $\kappa = \frac{f+2}{f}$\\
Isentropic (adiabatic) expansion $\frac{T_1}{T_0} = (\frac{p_1}{p_0})^\frac{\kappa-1}{\kappa}$\\
Ideal gas constant $R_m = 8.314 \frac{\text{kg }\text{m}^2}{\text{s }\text{mol }\text{K}}=R\cdot M$\\
\section{Chemical Equations}
Massfraction $\mu_i = \frac{m_i}{m}$\\
Molefraction $\nu_i = \frac{n_i}{n}$\\
Molemass $M_i = \frac{m_i}{n_i}$\\
Average mole mass $\bar{M} = \frac{m}{n} = \sum M_i \cdot \nu_i$\\
Mole to Mass fraction $\mu_i = \frac{M_i}{\bar{M}} \nu_i$\\
\vspace{3pt}
Oxygen balance $W_{ox} = -1600 (2x+y/2-z)/M$ with $C_xH_yO_z$\\
\section{Assumptions of NASA CEA}
Adiabatic combustion chamber.\\
Isentropic Expansion.\\
Homogenous mixing.\\
Thermochemical Equilibrium.\\
\vspace{5pt}
The \textbf{frozen} option of CEA will freeze all reaction products after the freezing point, allowing no further reactions in the mixture.

\section{Geometrics}
Sphere volume $V = \frac{4}{3}\pi R^3$\\
Sphere surface $A = 4\pi R^2$\\
\vspace{1pt}
Cylinder Side $A = 2\pi R\cdot h$

\section{Calculating solid/hybrid rocket fuels}
\begin{tabular}{lll}
	Pressure Exponent & $n$ & - \\
	Fuel Surface & $A_b$ & $\text{m}^2$ \\
	Port Area & $A_p$ & $\text{m}^2$ \\
	Fuel Density & $\rho_b$ & $\frac{\text{kg}}{\text{m}^3}$ \\
	Temperature Sensitivity Factor & $\Pi_{\dot{r}}$ & - \\
	Constants & $a_{ref}, T_{ref}$ & - \\
	Effective Burn Duration & $\Delta t_{AD}$ & s \\
\end{tabular}\\
\vspace{7pt}
Regressionrate $\dot{r}=a\cdot p_{c,0}^n = \frac{\dot{m}}{A_b\rho_b}$\\
with $a = a_{ref}\cdot e^{\Pi_{\dot{r}} (T-T_{ref})}$\\
Temp. Sensitivity $\Pi_{\dot{r}} = \frac{1}{\dot{r}} (\frac{d \dot{r}}{d T})_p$ \\
\vspace{5pt}
Regressionrate hybrid $\dot{r}=\frac{dR}{dt}=a\cdot G_{ox}^n = a\cdot (\frac{\dot{m}_{ox}}{A_p})^n$\\
\vspace{5pt}
If the pressure exponent $n$ is greater 1, the combustion chamber is sensible to pressure disturbances and can become in-stable. This leads to a destruction of the motor.\\
\vspace{7pt}
Combustion pressure $p_c = (c^* \rho_b a \frac{A_b}{A_t})^\frac{1}{1-n}$\\
with $ K = \frac{A_b}{A_t}$ as "clamping".\\
\vspace{5pt}
Charact. Speed $c^* = \frac{\sqrt{\frac{R_m}{\bar{M}}T_c}}{\Gamma}$\\
\vspace{5pt}
Gamma Function $\Gamma = \sqrt{\kappa \cdot (\frac{2}{\kappa+1})^{\frac{\kappa+1}{\kappa-1}}}$\\
\vspace{5pt}
Expansion Ratio $\epsilon = \frac{A_e}{A_t} = \Gamma \sqrt{\frac{\kappa-1}{2 \kappa}} \frac{(\frac{p_e}{p_c})^{-\frac{1}{\kappa}}}{\sqrt{1-(\frac{p_e}{p_c})^\frac{\kappa-1}{\kappa}}}$\\
\vspace{5pt}
Spec. Impulse $I_{sp} = \frac{1}{g} \left[ \sqrt{\frac{2\kappa}{\kappa-1}\frac{R_m}{\bar{M}} T_c \left[1-(\frac{p_e}{p_c})^\frac{\kappa-1}{\kappa}\right]} + \frac{(p_e-p_a) A_e}{\dot{m}}\right]$\\
\vspace{5pt}
Effective Thrust $F_{\text{eff}}=\int_{t_A}^{t_D}F\cdot dt / \Delta t_{AD}$\\
\vspace{5pt}

\section{Solid}
\begin{itemize}
	\item \textbf{Good}: High thrust density, storable, low structure mass, no pumps, cheap
	\item \textbf{Bad}: Short burn times, no reignition, moderate specific impulses, sensible to impacts, not throttable 
\end{itemize}

\section{Liquid}
\begin{itemize}
	\item \textbf{Good}: High specific impulses, reignitable, variable mass flow
	\item \textbf{Bad}: Movement of fuel in tanks, leakage, complex turbo pumps, big temperature range, zero-g fuel supply 
\end{itemize}

\section{Gel}
Gel propellants are mixtures containing at least two phases. A base fluid and a gelator to solidify the base fluid.
\vspace{7pt}
\begin{itemize}
	\item \textbf{Good}: easy to store like solids, can handle metal additives without sedimentation, in use similar to liquid propellants due to shear-thinning behaviour
	\item \textbf{Bad}: lower density and thrustdensity, lower specific impulses, higher operating pressures to transport the gel are necessary
\end{itemize}
\vspace{7pt}

\section{Hybrid}
Hybrid propellants are diffusion limited, which means that the regression rate is not pressure dependent.
\vspace{7pt}
\begin{itemize}
	\item \textbf{Good}: Inherently safe because no oxidizer contained in the fuel, can be throttled and reignited
	\item \textbf{Bad}: low regression rates, oxidizer-fuel shift in some fuels during operation, not much experience
\end{itemize}
\vspace{7pt}
It has been shown that Paraffin cores have a higher regression rate (3-5 times) than the classical HTPB cores.

\section{Green Propellants}
\begin{itemize}
	\item \textbf{Good}: Lower toxicity, less environmental hazards, lower costs
	\item \textbf{Bad}: medium specific impulses, not much experience
\end{itemize}

\section{Editorial}
Created by Christian Molli\`ere.\\
Last updated \today.\\
Feel free to share and edit!


\end{multicols}
\end{document}
